\newpage
\appendix
\section{Note teoriche}

Nella matematica classica una proposizione è vera o falsa
indipendentemente dal fatto che si possa provare o no.
%Da qui nascono tutti i casini

Lo stesso non si può dire per la matematica costruttiva, dove
(ad esempio) $A \lor \neg A$ non è valido a prescindere.

Per farlo occorre fornire una prova di A o una prova di $\neg A$.
Quindi $A \lor \neg A$ non è una tautologia.
La spiegazione costruttivista di una proposizione si dà in termini
di prove, non in termini di oggetti matematici che esistono
indipendentemente da tutto.

La frase ``propositions as sets'' identifica le proposizioni con
l'insieme delle loro prove.

Ad esempio, la proposizione $A \land B$ sarà un insieme dato
dalle sue prove. Una prova di $A \land B$ è una coppia dove il
primo elemento è una prova di A e il secondo una prova di B.

Quando due espressioni sono dei sinonimi da un punto di vista sintattico,
si dice che sono uguali definizionalmente o intensionalmente e si usa
il simbolo $\equiv$.
Ad esempio $e \equiv ((x)e)x$

Dal momento in cui si vedono le proposizioni come insiemi, il falso
corrisponde all'insieme vuoto.

Per ogni insieme A con operazioni ci sono 4 tipi di regole da definire:

\begin{itemize}
\item \textbf{Regole di formazione} - spiegano sotto quali assunzioni
A è un insieme e quando è uguale a un altro insieme B.
\item \textbf{Regole di introduzione} - spiegano come si formano gli
elementi canonici di A e quando due elemeni canonici sono uguali.
In queste regole vengono introdotti i costruttori di A.
\item \textbf{Regole di eliminazione} - mostrano come provare una
proposizione per un elemento arbitrario dell'insieme. Queste regole
usano un approccio induttivo in quanto basta dimostrare che un
elemento canonico arbitrario gode di una proprietà P per
dimostrare che vale anche per un altro elemento.
Il selettore/eliminatore viene introdotto in questa regola ed è una
costante non canonica del tipo che permette di effettuare pattern
matching e ricorsione primitiva.
\item \textbf{Regole di conversione} - descrivono le regole di
computazione del selettore/eliminatore introdotto nelle regole precedenti.
\end{itemize}
