\newpage
\section{Sull'uguaglianza proposizionale}

\subsection{Esercizio 1}

Per capire l'importanza della regola di eliminazione del tipo di uguaglianza
proposizionale formulato da Martin-Lof si dimostri che se si aggiunge alla
teoria dei tipi con il tipo di uguaglianza proposizionale formulato da
Martin-Lof le seguenti regole di unicità dei proof-term di uguaglianza:

\begin{prooftree}
\AxiomC{$a \in A [\gamma]$}
\AxiomC{$p \in Id_p(A, a, a) [\gamma]$}
\BinaryInfC{$Iduni(a, p) \in Id(Id(A, a, a), p, id(a)) [\gamma]$}
\end{prooftree}

\begin{prooftree}
\AxiomC{$a \in A [\gamma]$}
\UnaryInfC{$Iduni(a, id(a)) = id(id(a)) \in Id(Id(A, a, a) , id(a), id(a) )$}
\end{prooftree}

allora si dimostra che dati $a \in A \quad [\gamma]$ e $b \in A \quad [\gamma]$ c'è un
proof-term pf: \\

\begin{center}
$pf \in \forall_{w1 \in Id(A, a, b)} \forall_{w2 \in Id(A, a, b)} Id(Id(A, a, b), w1, w2) \quad [\gamma]$
\end{center}

ovvero i proof-term di un tipo di uguaglianza proposizionale sono tutti uguali
proposizionalmente.

\begin{lstlisting}[language=Coq]
Inductive MartinLof (A : Type) : A -> A -> Type :=
| ml_eq a: MartinLof A a a.

Lemma I_IdUni {A} (a : A) (p : MartinLof A a a) : MartinLof _ p (ml_eq A a).
Proof.
Admitted.

Lemma C_Iduni {A} (a: A):
MartinLof _ (I_IdUni a (ml_eq A a)) (ml_eq (MartinLof A a a) (ml_eq A a)).
Proof.
Admitted.

Lemma pf {A} (a b : A) : forall w1 : MartinLof A a b, (forall w2 : MartinLof A a b, MartinLof (MartinLof A a b) w1 w2).
Proof.
move=> w1 w2.
destruct w1.
by destruct (I_IdUni a w2).
Qed.
\end{lstlisting}

\subsection{Esercizio 3}

Sia k un numero naturale. Dare le regole del tipo $N_k$ con k elementi.
Poi dimostrare che nella teoria dei tipi estesa con i tipi $N_k$, ognuno di
questi tipi risulta isomorfo a un tipo definibile a partire dai costrutti di
tipo precedentemente descritti. \\

\textbf{Definizione di isomorfismo} \\

Un tipo $A \quad type \quad [\gamma]$ si dice isomorfo a un tipo $B \quad type \quad [\gamma]$
nella teoria dei tipi T se in T si possono derivare due termini: \\

$f(x) \in B \quad [\gamma, x \in A]$ e $h(y) \in A \quad [\gamma, y \in B]$ \\

tali per cui esistono dei proof-term pf1 e pf2 per cui in T si derivano: \\

$pf1 \in Id(A, x, h(f(x))) \quad [\gamma, x \in A]$ e
$pf2 \in Id(B, y, f(h(y))) \quad [\gamma, y \in B]$. \\

Sia k un numero naturale. \\

\textbf{Regole di formazione}: \\

\begin{prooftree}
\AxiomC{$\gamma$ cont.}
\UnaryInfC{$N_k \quad type \quad [\gamma]$}
\end{prooftree}

\textbf{Regole di introduzione}: \\

\begin{prooftree}
\AxiomC{$\gamma$ cont.}
\UnaryInfC{$0 \in N_k \quad [\gamma]$}
\end{prooftree}

\begin{prooftree}
\AxiomC{$\gamma$ cont.}
\UnaryInfC{$k-1 \in N_k \quad [\gamma]$}
\end{prooftree}

\textbf{Regole di eliminazione}: \\

\begin{prooftree}
\AxiomC{$k \in N_k \quad [\gamma]$}
\AxiomC{$D(z)$ type $[\gamma, z \in N_k]$}
\AxiomC{$d_0 \in D(0) \quad [\gamma]$}
\AxiomC{$d_{k-1} \in D(k-1) \quad [\gamma]$}
\QuaternaryInfC{$El_k(k, d_0, ..., d_{k-1}) \in D(k) \quad [\gamma]$}
\end{prooftree}

\textbf{Regole di conversione}: \\

nota: $i \in [0 ... k-1]$

\begin{prooftree}
\AxiomC{$k \in N_k \quad [\gamma]$}
\AxiomC{$D(z)$ type $[\gamma, z \in N_k]$}
\AxiomC{$d_0 \in D(0) \quad [\gamma]$}
\AxiomC{$d_{k-1} \in D(k-1) \quad [\gamma]$}
\QuaternaryInfC{$El_k(k, d_0, ..., d_{k-1}) = c_i \in D(k) \quad [\gamma]$}
\end{prooftree}

Preso k = 2, mostrare l'isomorfismo tra $N_2$ e $\{x|x \leq 2\}$ (in Coq):

\begin{lstlisting}[language=Coq]
Inductive N2 := zero | uno.

Definition minore_di_due := {x|x<2}.

Lemma zero_min_due : 0<2.
Proof.
by [].
Qed.

Lemma uno_min_due : 1<2.
Proof.
by [].
Qed.

Definition f (x : N2) : minore_di_due :=
 match x with
 | zero => exist _ 0 zero_min_due
 | uno => exist _ 1 uno_min_due
 end.

Definition h (y : minore_di_due) : N2 :=
 match y with
 | exist 0 _ => zero
 | exist (S n) _ => uno
 end.

Lemma pf1 (x : N2) : x = h (f x).
Proof.
by case x.
Qed.

Lemma ex_falso P : False -> P.
Proof. move=> abs. case: abs. Qed.

Lemma pf2 (y : minore_di_due) : y = f (h y).
Proof.
apply: val_inj.
case: y => [[| [|y]] Py].
by rewrite /=.
by rewrite /=.
rewrite /=.
by apply: ex_falso.
Qed.
\end{lstlisting}
