\newpage

\section{Breve note su Coq}

Cosa importare per svolgere gli esercizi con il proof assistant?

\begin{lstlisting}[language=Coq]
From Coq Require Import Init.Prelude Unicode.Utf8.
From Coq Require Import ssreflect ssrfun ssrbool.
From mathcomp Require Import eqtype ssrnat div prime.
\end{lstlisting}

\subsection{Comandi ricorrenti e loro breve descrizione}

Una lista di comandi ricorrenti negli esercizi svolti:

\begin{itemize}
\item \textbf{Match}: fornisce l'analisi per casi. I casi del match parlano
della forma canonica di un termine.
 
\item \textbf{Fixpoint}: permette di scrivere definizioni induttive.

Le regole computazionali dell'eliminatore sono date dalla combinazione di match
e Fixpoint

\item \textbf{Funzioni anonime}, esempio: \texttt{fun x : nat => x + 1}

\item \textbf{Tipi di dato}, esempio: \texttt{Inductive nat := O | S (n : nat)}

\item \textbf{Pattern matching}, esempio: \texttt{match x with O => .. | S p => ..}

\item \textbf{Ricorsione}, esempio: \texttt{Fixpoint addn n m := ...}

\item \textbf{Polimorfismo}, esempio: \texttt{Inductive pair (A B : Type) := ..}

\item \textbf{Apply t}: Questa tattica si applica a qualsiasi goal e tenta di fare un
match con la conclusione del termine t che riceve in input.
In altre parole, usa t per dimostrare il goal corrente, ovvero t è un termine
che abita il tipo del goal.

\item \textbf{Dimostrare} = costruire termini passo passo.

\item \textbf{Struttura di un lemma}: \texttt{Lemma nome : tipo. Proof. ...prova... Qed.}

\item \textbf{Move}: La tattica $=>$ sposta le variabili, le definizioni locali e le
assunzioni nel contesto corrente.

\item \textbf{Unfold}: Sostituisce il nome con il corpo: \texttt{unfold name in hyp}.

\item \textbf{Rewrite}: riscrive un'espressione usando le uguaglianze a disposizione (sia nelle
librerie che quelle precedentemente definite).

\item \textbf{Erefl}: dimostra eq per termini uguali una volta nomalizzati.

\item \textbf{Forall}: lega tipi, termini o niente. Per dimostrare un predicato del tipo
$\forall x : A, P x$ occorre fornire una funzione che, data una prova di A (chiamata x),
produce una prova di P x.

\item \textbf{Change}: cambia la forma del goal corrente in una convertibile.

\item \textbf{Case}: effettua un'analisi per casi senza ricorsione.

\item \textbf{Elim}: effettua un'analisi per induzione.
\end{itemize}

\newpage
\section{Tipi singoletto, delle liste e dei numeri naturali}

\subsection{Esercizio 1-A}

Dati i tipi singoletto e delle liste è possibile definire un tipo dei numeri 
naturali Nat? \\

Sì, ciascun numero n è rappresentabile come una lista di n singoletti (nil
corrisponderà allo zero). Ad esempio $1 := cons(nil, *)$.

\subsection{Esercizio 1-B}

La teoria dei tipi che include tutte le regole finora descritte (singoletto,
liste e naturali) eccetto quelle di sostituzione e di indebolimento rende
ammissibili le regole di sostituzione e anche quelle di indebolimento.

Qui di seguito viene solo mostrato qualche esempio che verifica la precedente
affermazione.

Prima regola di indebolimento:

\begin{prooftree}
\AxiomC{A type $[\Gamma]$}
\AxiomC{$\Gamma, \Delta$  cont.}
\BinaryInfC{A type $[\Gamma, \Delta]$}
\end{prooftree}

Esempio per il tipo singoletto:

\begin{prooftree}
\AxiomC{N1 type $[\Gamma]$}
\AxiomC{$\Gamma, \Delta$ cont.}
\BinaryInfC{N1 type $[\Gamma, \Delta]$}
\end{prooftree}

La regole è ammissibile, infatti esiste la seguente derivazione:

\begin{prooftree}
\AxiomC{$\Gamma, \Delta$ cont.}
\UnaryInfC{N1 type $[\Gamma, \Delta]$}
\end{prooftree}

Lo stesso caso si applica in modo simile per i tipi List e Nat.

\subsection{Esercizio 2}

Definire l'operazione \textbf{append} di accostamento di una lista a un'altra di un tipo A 
type [ ] usando le regole del tipo delle liste:
$append(x, y) \in List(A) \quad [x \in List(A), y \in List(A)]$
in modo tale che valga $append(x, nil) = x \in List(A) \quad [x \in List(A)]$

Nota: definisco una funzione ausiliaria appendEnd che aggiunge un elemento alla fine 
di una lista (al contrario di cons).

\begin{lstlisting}[language=Coq]
(* Definizione del tipo lista *)
Inductive list (A : Type) :=
  nil | cons (a : A) (tl : list A).

Arguments nil {_}.
Arguments cons {_} a tl.

Infix "::" := cons.

Fixpoint appendEnd {A} (x : list A) (y : A) :=
  match x with
  | nil => y :: nil
  | s :: c => s :: (appendEnd c y)
  end.

Fixpoint append {A} (x : list A) (y : list A) :=
  match y with
  | nil => x
  | a::bs => append (appendEnd x a) bs
  end.
\end{lstlisting}

\subsection{Esercizio 3}

I seguenti esercizi richiedono di scrivere funzioni che accettano tutte una 
lista non vuota (diversa da nil). Viene quindi preliminarmente definito un tipo 
notNil:

\begin{lstlisting}[language=Coq]
Definition notNil (A : Type) := { lt : list A | ~lt = nil }.
\end{lstlisting}

Specificare il tipo e definire le operazioni: \\

(a) \textbf{back}, che data una lista non vuota ne estrae la lista meno il primo elemento:

\begin{lstlisting}[language=Coq]
Lemma ex_falso P : False -> P.
Proof. move=> abs. case: abs. Qed.

Definition back {A} (x : notNil A) : list A.
Proof.
case: x.
move=> lt.
case: lt.
  move=> abs.
  apply: ex_falso.
  apply: abs.
  apply: erefl.
move=> _ lt _.
apply: lt.
Defined.
\end{lstlisting}

\begin{lstlisting}[language=Coq]
(* back accetta argomenti di tipo notNil *)

Lemma not_123_nil : ~ ( (1 :: 2 :: 3 :: nil) = nil).
Proof.
move=> E.
change (match (1 :: 2 :: 3 :: nil) with nil => True | (_ :: _) => False end).
rewrite E.
apply: I.
Qed.

Definition n123n : notNil nat := exist _ (1 :: 2 :: 3 :: nil) not_123_nil.

Eval compute in back n123n.
\end{lstlisting}

(b) l'operazione \textbf{front} che data una lista non vuota ne estrae la lista meno
l'ultimo elemento

da completare

\begin{lstlisting}[language=Coq]
Fixpoint front {A} (x : list A) : list A :=
  match x with
  | nil => nil
  | a :: nil => nil
  | s :: c => s :: front c
  end.
\end{lstlisting}

(c) l'operazione \textbf{last} che data una lista non vuota ne estrae l'ultimo elemento

(d) l'operazione \textbf{first} che data una lista non vuota ne estrae il primo
elemento

\begin{lstlisting}[language=Coq]
Definition first {A} (x : notNil A) : A.
Proof.
case: x.
move=> lt.
case: lt.
  move=> abs.
  apply: ex_falso.
  apply: abs.
  apply: erefl.
move=> a _ _.
apply: a.
Defined.

Eval compute in back n123n.
\end{lstlisting}

L'operazione first è definita dal seguente elemento (non canonico e non in
forma normale): $first(x') = El_{list}(x',nil,(x,y,z).y)$

\subsection{Esercizio 4}

Definire un'operazione binaria \textbf{bin(x, y)} usando le regole del tipo dei numeri
naturali: $bin(x, y) \in Nat \quad [x \in Nat, y \in Nat]$
in modo tale che valga $bin(x, 0) = 0 \in Nat \quad [x \in Nat].$

\begin{lstlisting}[language=Coq]
Definition bin (x : nat)  (y : nat) :=
  match y with
  | 0 => 0
  | S c => x
  end.

Lemma ex_bin (x : nat) : bin x 0 = 0.
Proof.
apply: erefl.
Qed.
\end{lstlisting}

\subsection{Esercizio 5}

Definire l'operazione di addizione usando le regole del tipo dei numeri naturali:
$x + y \in Nat \quad [x \in Nat, y \in Nat]$
in modo tale che valga $x + 0 = x \in Nat \quad [x \in Nat]$.

\begin{lstlisting}[language=Coq]
Fixpoint sum (x : nat) (y : nat) :=
  match y with
  | 0 => x
  | S c => S (sum x c)
  end.

Lemma ex_sum (x : nat) : sum x 0 = x.
Proof.
apply: erefl.
Qed.
\end{lstlisting}

\subsection{Esercizio 6}

Definire l'operazione di addizione usando le regole del tipo dei numeri naturali:
$x + y \in Nat \quad [x \in Nat, y \in Nat]$
in modo tale che valga $0 + x = x \in Nat \quad [x \in Nat]$.

\begin{lstlisting}[language=Coq]
Fixpoint sum (x : nat) (y : nat) :=
  match x with
  | 0 => y
  | S c => S (sum c y)
  end.

Lemma ex_sum (x : nat) : sum 0 x = x.
Proof.
apply: erefl.
Qed.
\end{lstlisting}

L'operazione sum è definita dal seguente elemento (non canonico e non in
forma normale): $sum(x',y') = El_{Nat}(x',y',(x,y).succ(y))$

\subsection{Esercizio 7}

Definire l'operazione di moltiplicazione usando le regole del tipo dei numeri 
naturali: $x*y \in Nat \quad [x \in Nat, y \in Nat]$
in modo tale che valga $x*0 = 0 \in Nat \quad [x \in Nat]$.

\begin{lstlisting}[language=Coq]
Fixpoint mul (x : nat) (y : nat) :=
  match y with
  | 0 => 0
  | 1 => x
  | S c => x + (mul x c)
  end.

Lemma mul_0 (x : nat) : mul x 0 = 0.
Proof.
apply: erefl.
Qed.
\end{lstlisting}

L'operazione mul è definita dal seguente elemento (non canonico e non in
forma normale): $mul(x',y') = El_{Nat}(y',0,(x,y).sum(x,y))$
