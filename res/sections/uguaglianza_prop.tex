\newpage
\section{Uguaglianza proposizionale alla Gentzen/Lawvere e alla Leibniz}

\subsection{Esercizio 1}

In che relazione stanno i due tipi di uguaglianza?

Il secondo tipo di uguaglianza alla Leibniz è più flessibile rispetto al primo:
l'elemento $c(x) \in C(x,x) \quad [\gamma, x \in A]$ può dipendere da un elemento
$x \in A$ diverso dall'elemento $a \in A \quad [\gamma]$.

\subsection{Esercizio 2}

Con quale tipo di uguaglianza si può dimostrare che dato un termine
$f(x) \in B \quad [x \in A]$ si può trovare un proof-term q tale che:
$q \in Id(B, f(x), f(y)) \quad [x \in A, y \in A, w \in Id(A,x,y)]$?

\begin{lstlisting}[language=Coq]
Module Leibniz.

Definition eq {A} (a b : A) : Prop :=
  forall P : A -> Prop, P a -> P b.

Definition id {A} (a : A) : eq a a :=
  fun P a => a.

Definition el {A} (C : A -> A -> Prop) (a b : A) (p : eq a b) (c : forall x, C x x) : C a b :=
  p (C a) (c a).

Definition conv {A} (C : A -> A -> Prop) (a : A) (c : forall x, C x x) : C a a :=
  el C a a (id a) c.

End Leibniz.

Lemma q {A} {B} (x y : A) (f : A -> B) (w : Leibniz.eq x y) : (Leibniz.eq (f x) (f y)).
Proof.
apply (Leibniz.el (fun a b => Leibniz.eq (f a) (f b)) x y w (fun x => Leibniz.id (f x))).
Qed.
\end{lstlisting}

\subsection{Esercizio 3}

Con quale tipo di uguaglianza si può dimostrare che esiste un proof-term pf del
tipo: $pf \in P(y) \quad [x \in A, y \in A, z \in P(x), w \in Id(A,x,y)]$?

Si fa sempre riferimento al modulo dell'uguaglianza di Leibniz.

\begin{lstlisting}[language=Coq]
Lemma pf {A} {P} (x y : A) (z : P x) (w : Leibniz.eq x y) : Leibniz.eq y y.
Proof.
apply Leibniz.id.
Qed.
\end{lstlisting}

\subsection{Esercizio 4}

Con quale tipo di uguaglianza si può dimostrare che esiste un proof-term pf del
tipo: $pf \in Id(A,x,z) \quad [x \in A, y \in A, z \in A, w1 \in Id(A,x,y), w2 \in Id(A,y,z)]$?

\begin{lstlisting}[language=Coq]
Lemma pf {A} (x y z : A) (w1 : Leibniz.eq x y) (w2 : Leibniz.eq y z) : Leibniz.eq x z.
Proof.
apply (Leibniz.el (fun a b => Leibniz.eq a b) y z w2 (fun a => Leibniz.id a)).
apply w1.
Qed.
\end{lstlisting}

\subsection{Esercizio 5}

Con quale tipo di uguaglianza si può dimostrare che esiste un proof-term pf del
tipo: $pf \in Id(N_1, x, *) \quad [x \in N_1]$?

\begin{lstlisting}[language=Coq]
Lemma pf (x : unit) : Leibniz.eq x tt.
Proof.
case: x.
by [].
Qed.
\end{lstlisting}

\begin{prooftree}
\AxiomC{$x \in N1$ $[\Gamma]$}
\AxiomC{$Id_L(N_1,z,*)$ type $[\Gamma,z \in N_1]$}
\AxiomC{$c \in Id_L(N_1, *, *)$ $[\Gamma]$}
\TrinaryInfC{$El_{N_1}(x,c) \in Id_L(N_1, x, *)$ $[\Gamma]$}
\end{prooftree}

dove:

\begin{prooftree}
\AxiomC{\checkmark}
\UnaryInfC{$N_1$ type $[\Gamma, z \in N_1]$}
\AxiomC{\checkmark}
\UnaryInfC{$z \in N_1$ $[\Gamma,z \in N_1]$}
\AxiomC{\checkmark}
\UnaryInfC{$* \in N_1$ $[\Gamma, z \in N_1]$}
\TrinaryInfC{$Id_L(N_1,z,*)$ type $[\Gamma, z \in N_1]$}
\end{prooftree}

e:

\begin{prooftree}
\AxiomC{\checkmark}
\UnaryInfC{$* \in N_1$ $[\Gamma]$}
\UnaryInfC{$c \in Id_L(N_1, *, *)$ $[\Gamma]$}
\end{prooftree}

c = id(x).

Quindi il proof-term cercato è:

$El_{N_1}(x,id(*)) \in Id_L(N_1, x, *)$ $[\Gamma]$

\subsection{Esercizio 6}

Con quale tipo di uguaglianza si può dimostrare che esiste un proof-term tale
che è possibile definire l'addizione tra numeri naturali:
$x + y \in Nat \quad [x \in Nat, y \in Nat]$ in modo tale che esistano dei proof-term pf1 e pf2 tali che: $pf1 \in Id(Nat, x + 0, x) \quad [x \in Nat], pf2 \in Id(Nat, 0 + x, x) \quad [x \in Nat]$?

\begin{lstlisting}[language=Coq]
Lemma pf1 (x : nat) : Leibniz.eq (x + 0) x.
Proof.
elim: x.
by [].
move=> x IHp //=.
apply (Leibniz.el (fun a b => Leibniz.eq b a) (x + 0) x IHp (fun a => Leibniz.id a)).
apply Leibniz.id.
Qed.

Lemma pf2 (x : nat) : Leibniz.eq (0 + x) x.
Proof.
case: x.
by [].
move=> n IHp //=.
Qed.
\end{lstlisting}

\subsection{Esercizio 8}

Si dimostri che esiste un proof-term pf tale che:
$pf \in Id(Nat, x' +_1 y', x' +_2 y') \quad [x' \in Nat, y' \in Nat]$
dove $x' +_1 y' = sum1$ e $x' +_2 y' = sum2$.

\begin{lstlisting}[language=Coq]
Fixpoint sum1 (x : nat) (y : nat) :=
  match y with
  | 0 => x
  | S c => S (sum1 x c)
  end.

Fixpoint sum2 (x : nat) (y : nat) :=
  match x with
  | 0 => y
  | S c => S (sum2 c y)
  end.

Lemma sum1_zero (x : nat) : sum1 0 x = x.
Proof.
elim x.
by [].
move=> n IHp //=.
by rewrite IHp.
Qed.

Lemma sum2_zero (x : nat) : sum2 x 0 = x.
Proof.
elim x.
by [].
move=> n IHp //=.
by rewrite IHp.
Qed.

Lemma sum2_succ (x y : nat) : sum2 x (S y) = S (sum2 x y).
Proof.
elim x.
by [].
move=> n IHp //=.
by rewrite IHp.
Qed.

Lemma pf (x y : nat) : Leibniz.eq (sum1 x y) (sum2 x y).
Proof.
elim y.
by rewrite /= sum2_zero.
move=> n IHp //=.
rewrite sum2_succ.
apply (Leibniz.el (fun a b => Leibniz.eq a b) (sum1 x n) (sum2 x n) IHp (fun a => Leibniz.id a)).
apply: Leibniz.id.
Qed.
\end{lstlisting}

\subsection{Esercizio 10}

Per esercizio si provi a definire il tipo dell'insieme ``vuoto'' N0.

\begin{lstlisting}[language=Coq]
Inductive N0 : Set := .
\end{lstlisting}
