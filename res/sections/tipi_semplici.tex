\newpage
\section{Rappresentazione dei tipi semplici in teoria dei tipi dipendenti}

\subsection{Esercizio 1}

Si dimostri che se $a = b \in A \quad [\gamma]$ è derivabile nella teoria dei tipi
con le regole finora introdotte esiste un proof-term pf tale che:
$pf \in Id(A, a, b) \quad [\gamma]$ è derivabile.

\begin{lstlisting}[language=Coq]
Lemma a_equal_b {A} (a b : A) : a = b.
Proof.
Admitted.

Lemma pf {A} (a b : A) : eq a b.
Proof.
apply (a_equal_b a b).
Qed.
\end{lstlisting}

Premessa: se $a = b \in A \quad [\gamma]$ è derivabile allora anche
$A$ type $[\gamma]$, $a \in A$ $[\gamma]$ e $b \in A$ $[\gamma]$
sono derivabili.\\

Usando la regola \textit{sub-eq-typ}:

\begin{prooftree}
\AxiomC{\checkmark}
\UnaryInfC{\dots}
\UnaryInfC{$Id(A,x,y)$ type $[\Gamma, x \in A, y \in A]$}
\AxiomC{\checkmark}
\UnaryInfC{\dots}
\UnaryInfC{$a = a \in A \quad [\Gamma]$}
\AxiomC{\checkmark}
\UnaryInfC{$a = b \in A \quad [\Gamma]$}
\TrinaryInfC{$Id(A,a,a) = Id(A,a,b)\quad [\Gamma]$}
\end{prooftree}

e poi la regola \textit{conv}:

\begin{prooftree}
\AxiomC{\checkmark}
\UnaryInfC{\dots}
\UnaryInfC{$id(a) \in Id(A,a,a) \quad [\Gamma]$}
\AxiomC{\checkmark}
\UnaryInfC{$Id(A,a,a) = Id(A,a,b)\quad [\Gamma]$}
\BinaryInfC{$id(a) \in Id(A,a,b)\quad [\Gamma]$}
\end{prooftree}

la derivazione è conclusa.

\subsection{Esercizio 2A}

La $\eta$-conversione del tipo prodotto ovvero:
$<\pi_1(z),\pi_2(z)> = z \in A X B \quad [z \in A X B]$
è derivabile?

Non è derivabile. Il termine $<\pi_1(z),\pi_2(z)>$ è in forma normale e canonica,
mentre la variabile z è in forma normale ma non è canonica.
Se due termini sono uguali definizionalmente, allora anche le loro forme normali
devono essere identiche e in questo caso è falso.

\subsection{Esercizio 2B}

\'{E} derivabile la corrispondente uguaglianza proposizionale, ovvero esiste un
proof-term pf tale che:
$pf \in Id(A X B, <\pi_1(z), \pi_2(z)>, z ) \in A X B \quad [z \in A X B]$ è
derivabile nella teoria dei tipi dipendenti?

\begin{lstlisting}[language=Coq]
Lemma pf {A} {B} (z : prod A B) : eq (pair (fst z) (snd z)) z.
Proof.
destruct z.
by rewrite /=.
Qed.
\end{lstlisting}
