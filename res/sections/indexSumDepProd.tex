\newpage
\section{Tipi somma indiciata e tipi prodotto dipendente}

\subsection{Esercizio 1}

Provare a scrivere le regole del tipo prodotto cartesiano AxB di un tipo A con
un tipo B. \\

\textbf{Regola di formazione:}

\begin{prooftree}
\AxiomC{A type $[\gamma]$}
\AxiomC{B type $[\gamma]$}
\BinaryInfC{AxB type $[\gamma]$}
\end{prooftree}

\textbf{Regola di introduzione:}

\begin{prooftree}
\AxiomC{$a \in A \quad [\gamma]$}
\AxiomC{$b \in B \quad [\gamma]$}
\BinaryInfC{<a,b> $\in$ AxB $[\gamma]$}
\end{prooftree}

\textbf{Regole di eliminazione:}

\begin{prooftree}
\AxiomC{$d \in$ AxB $[\gamma]$}
\UnaryInfC{$\pi_1(d) \in A [\gamma]$}
\end{prooftree}

\begin{prooftree}
\AxiomC{$d \in$ AxB $[\gamma]$}
\UnaryInfC{$\pi_2(d) \in B [\gamma]$}
\end{prooftree}

\textbf{Regole di conversione:}

\begin{prooftree}
\AxiomC{$a \in A \quad [\gamma]$}
\AxiomC{$b \in B \quad [\gamma]$}
\BinaryInfC{$\pi_1(<a,b>) = a \in A [\gamma]$}
\end{prooftree}

\begin{prooftree}
\AxiomC{$a \in A \quad [\gamma]$}
\AxiomC{$b \in B \quad [\gamma]$}
\BinaryInfC{$\pi_2(<a,b>) = b \in B [\gamma]$}
\end{prooftree}

\subsection{Esercizio 2}

Provare a scrivere le regole del tipo delle funzioni $A \rightarrow B \quad [\gamma]$ da un
tipo A a un tipo B. \\

\textbf{Regola di formazione:}

\begin{prooftree}
\AxiomC{A type $[\gamma]$}
\AxiomC{B type $[\gamma]$}
\BinaryInfC{$A \rightarrow B$ type $[\gamma]$}
\end{prooftree}

\textbf{Regola di introduzione:}

\begin{prooftree}
\AxiomC{$f(x) \in B \quad [\gamma, x \in A]$}
\UnaryInfC{$\lambda x. f(x) \in A \rightarrow B \quad [\gamma]$}
\end{prooftree}

\textbf{Regola di eliminazione:}

\begin{prooftree}
\AxiomC{$a \in$ A $[\gamma]$}
\AxiomC{$f \in A \rightarrow B \quad [\gamma]$}
\BinaryInfC{$Ap(f, a) \in B(a) \quad [\gamma]$}
\end{prooftree}

\textbf{Regola di conversione:}

\begin{prooftree}
\AxiomC{$a \in A \quad [\gamma]$}
\AxiomC{$f(x) \in B \quad [\gamma, x \in A]$}
\BinaryInfC{$Ap(\lambda x. f(x), a) = f(a) \in B [\gamma]$}
\end{prooftree}

\subsection{Esercizio 3}

Come rappresentare con i tipi finora descritti il tipo dei naturali positivi?
E il tipo delle liste non vuote?

\begin{lstlisting}[language=Coq]
Definition nat_pos := { n : nat | ~ n = 0 }.

Lemma uno_not_zero : ~ (1 = 0).
Proof.
by [].
Qed.

Definition uno_pos : nat_pos := exist _ 1 uno_not_zero.

Inductive list (A : Type) :=
  nil | cons (a : A) (tl : list A).

Arguments nil {_}.
Arguments cons {_} a tl.

Definition list_notNil {A} := { l : list A | ~ l = nil }.

Infix "::" := cons.

Lemma list_not_nil : ~ (1 :: nil = nil).
Proof.
by [].
Qed.

Definition uno_not_nil : list_notNil := exist _ (1 :: nil) list_not_nil.

\end{lstlisting}

\subsection{Esercizio 4}

Come rappresentare con i tipi finora descritti il tipo delle funzioni tra
naturali positivi?

Si rappresenta con il tipo $\Pi_{x \in Nat+} Nat+$.

\begin{lstlisting}[language=Coq]
Definition funNatPos := nat_pos -> nat_pos.
\end{lstlisting}
