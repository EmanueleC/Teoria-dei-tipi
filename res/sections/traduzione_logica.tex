\newpage
\section{Come tradurre la logica predicativa con uguaglianza in teoria dei tipi}

\subsection{Esercizio 1}

Si mostri che il seguente principio di scelta, detto comunemente assioma di scelta è derivabile: \\

$\forall x \in A, \exists y \in B \quad \Re (x, y) \rightarrow \exists f \in A \rightarrow B, \forall x \in A \quad \Re (x, f(x)) \quad [\gamma]$ \\

Traduzione logica: \\

Occorre trovare un proof-term pf: \\

$pf \equiv \lambda w. p(w) \in \prod_{x \in A} \sum_{y \in B} \Re (x, y) \rightarrow \sum_{f \in A \rightarrow B} \prod_{x \in A} \Re (x, f(x)) \quad [\gamma]$ \\

Applicazione della regola di introduzione di $\rightarrow$:

\begin{prooftree}
\AxiomC{$p(w) \in \sum_{f \in A \rightarrow B} \prod_{x \in A} \Re (x, f(x)) \quad [\gamma, w \in \prod_{x \in A} \sum_{y \in B} \Re (x, y)]$}
\UnaryInfC{$\lambda w. p(w) \in \prod_{x \in A} \sum_{y \in B} \Re (x, y) \rightarrow \sum_{f \in A \rightarrow B} \prod_{x \in A} \Re (x, f(x)) \quad [\gamma]$}
\end{prooftree}

Si cerca il termine $p(w)$.

Applicando la regola di introduzione di $\sum$:

\begin{prooftree}
\alwaysNoLine
\AxiomC{}
\UnaryInfC{$m \in A \rightarrow B \quad [\gamma, w \in \prod_{x \in A} \sum_{y \in B} \Re (x, y)]$}
\UnaryInfC{$n \in \prod_{x \in A} \Re (x, \pi_1 (Ap (w, x)) \quad [\gamma, w \in \prod_{x \in A} \sum_{y \in B} \Re (x, y)]$}
\UnaryInfC{$\prod_{x \in A} \Re (x, f(x)) type \quad [\gamma, w \in \prod_{x \in A} \sum_{y \in B} \Re (x, y)]$}
\alwaysSingleLine
\UnaryInfC{$<m,n> \in \sum_{f \in A \rightarrow B} \prod_{x \in A} \Re (x, f(x)) \quad [\gamma, w \in \prod_{x \in A} \Re (x, f(x))]$}
\end{prooftree}

m e n sono due termini funzione del tipo $\lambda x^A.b(x)$:

Applicazione della (prima) regola di eliminazione del prodotto cartesiano $X$:

\begin{prooftree}
\AxiomC{$Ap(w, x) \in \sum_{y \in B} \Re (x, y) \quad [\gamma, w \in \prod_{x \in A} \sum_{y \in B} \Re (x, y)]$}
\UnaryInfC{$\pi_1 (Ap(w, x)) \in A \rightarrow B \quad [\gamma, w \in \prod_{x \in A} \sum_{y \in B} \Re (x, y)]$}
\end{prooftree}

Applicazione della (seconda) regola di eliminazione del prodotto cartesiano $X$:

\begin{prooftree}
\AxiomC{$Ap(w, x) \in \sum_{y \in B} \Re (x, y) \quad [\gamma, w \in \prod_{x \in A} \sum_{y \in B} \Re (x, y)]$}
\UnaryInfC{$\pi_2 (Ap(w, x)) \in \Re (x, \pi_1 (Ap (w, x)) \quad [\gamma, w \in \prod_{x \in A} \sum_{y \in B} \Re (x, y)]$}
\end{prooftree}

Applicazione della regola di eliminazione di $\prod$:

\begin{prooftree}
\AxiomC{$x \in A \quad [\gamma, w \in \prod_{x \in A} \sum_{y \in B} \Re (x, y)]$}
\AxiomC{\checkmark}
\UnaryInfC{$w \in \prod_{x \in A} \sum_{y \in B} \Re (x, y) \quad [\gamma, w \in \prod_{x \in A} \sum_{y \in B} \Re (x, y)]$}
\BinaryInfC{$Ap(w, x) \in \sum_{y \in B} \Re (x, y) \quad [\gamma, w \in \prod_{x \in A} \sum_{y \in B} \Re (x, y)]$}
\end{prooftree}

Infine, applicando la regola di introduzione di $\sum$:

\begin{prooftree}
\alwaysNoLine
\AxiomC{}
\UnaryInfC{$\pi_1 (Ap(w, x)) \in A \rightarrow B \quad [\gamma, w \in \prod_{x \in A} \sum_{y \in B} \Re (x, y)]$}
\UnaryInfC{$\pi_2 (Ap(w, x)) \in \Re (x, \pi_1 (Ap (w, x)) \quad [\gamma, w \in \prod_{x \in A} \sum_{y \in B} \Re (x, y)]$}
\UnaryInfC{$\prod_{x \in A} \Re (x, f(x)) type \quad [\gamma, w \in \prod_{x \in A} \sum_{y \in B} \Re (x, y)]$}
\alwaysSingleLine
\UnaryInfC{$<\lambda x. \pi_1 (Ap(w, x)), \lambda x. \pi_2 (Ap(w, x))> \in \sum_{f \in A \rightarrow B} \prod_{x \in A} \Re (x, f(x)) \quad [\gamma, w \in \prod_{x \in A} \Re (x, f(x))]$}
\end{prooftree}

così, $p(w) =  <\lambda x. \pi_1 (Ap(w, x)), \lambda x. \pi_2 (Ap(w, x))>$.

\subsection{Esercizio 3A}

Supposto di rappresentare il tipo dei booleani come $Boole = N1 + N1$ con:
$0 = inl(*), 1 = inr(*)$

e supposto che esista un proof-term:
$dis \in N_0 \quad [z \in Id(Boole, 0, 1)]$

si dimostri che dati due tipi A type e B type esiste una funzione:
$f(z) \in \sum_{x \in Boole} ( (Id(Boole, x, 1) \rightarrow A) X (Id(Boole, x, 0) \rightarrow B) ) \quad [ z \in A + B ]$

\begin{lstlisting}[language=Coq]
Inductive Boole :=
  | inr (a : unit)
  | inl (a : unit).

Inductive N0 :=.

Lemma dis (z : eq (inr tt) (inl tt)) : N0.
Proof.
Admitted.

Lemma es {A} {B} (z: A + B) :
sigT (fun x : Boole => prod (eq x (inr tt) -> A) (eq x (inl tt) -> B)).
Proof.
case: z.
move=> a.
exists (inr tt).
rewrite //=.
move=> b.
exists (inl tt).
rewrite //=.
Qed.
\end{lstlisting}

\subsection{Esercizio 3B}

\dots ed anche una funzione:

$g(w) \in A + B \quad [w \in \sum_{x \in Boole} ( (Id(Boole, x, 1) \rightarrow A) x (Id(Boole, x, 0) \rightarrow B) )]$

\begin{lstlisting}[language=Coq]
Lemma eq_unit_inr_tt (a : unit) : eq (inr a) (inr tt).
Proof.
by case a.
Qed.

Lemma eq_unit_inl_tt (a : unit) : eq (inl a) (inl tt).
Proof.
by case a.
Qed.

Lemma es2 {A} {B} (w : sigT (fun x : Boole => prod (eq x (inr tt) -> A) (eq x (inl tt) -> B))) : A + B.
Proof.
destruct w.
destruct p.
case x.
\end{lstlisting}

\subsection{Esercizio 8}

Per ogni connettivo e quantificatore della logica predicativa con uguaglianza si
descrivano le relative regole con giudizi del tipo $\phi \quad prop \quad [\gamma]$ e
$\phi \quad true \quad [\gamma]$ di formazione, introduzione ed eliminazione che risultano
ammissibili in teorie dei tipi.

Svolto su carta.

\subsection{Esercizio 9}

Verificare che tutti gli assiomi di Peano sono validi in teoria dei tipi con le
regole finora introdotte e interpretando le formule logiche come sopra
descritto.

\begin{lstlisting}[language=Coq]
Lemma P1 : forall x : nat, 0 <> S x.
Proof.
by [].
Qed.

Definition pred (x : nat) : nat :=
  match x with
  | 0 => 0
  | S n => n
  end.

Lemma P2 : forall x y : nat, x = y -> S x = S y.
Proof.
move=> x y eq.
replace (x = y) with (pred (S x) = pred (S y)) by auto using pred_Sn.
by rewrite eq.
Qed.

Lemma P3 (x : nat) : x + 0 = x.
Proof.
by [].
Qed.

Lemma P4 : forall x y : nat, x + S y = S (x + y).
Proof.
move=> x y.
elim x.
by rewrite /=.
move=> n IHp.
by rewrite /= IHp.
Qed.

Lemma P5 : forall x : nat, x*0 = 0.
Proof.
by [].
Qed.

Lemma P6 : forall x y : nat, x*(S y) = x*y + x.
Proof.
by [].
Qed.

Lemma P7 : forall (x : nat) (P : nat -> Prop), P 0 -> (forall n : nat, P n -> P (S n)) -> P x.
Proof.
move=> x P base IHp.
elim x.
apply base.
move=> n Pn.
by apply: IHp.
Qed.
\end{lstlisting}
