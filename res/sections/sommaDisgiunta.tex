\newpage
\section{Tipo della somma disgiunta}

\subsection{Esercizio 1}

Si scrivano le regole del tipo booleano come tipo semplice e si provi che è
rappresentabile da N1 + N1.

\begin{lstlisting}[language=Coq]
Inductive Bool :=
  | inl (a : unit)
  | inr (b : unit)

Definition true : Bool := inl tt.
Definition false : Bool := inr tt.
\end{lstlisting}

\subsection{Esercizio 2}

Si definisca la funzione predecessore: $p(x) \in Nat \quad [x \in Nat]$
tale che $p(0) = 0$ e $p(succ(n)) = n$.

\begin{lstlisting}[language=Coq]
Fixpoint pred (x : nat) : nat :=
  match x with
  | 0 => 0
  | S x => x
  end.

Lemma pred_0 : pred 0 = 0.
Proof.
apply: erefl.
Qed.

Lemma pred_S (x : nat) : pred (S x) = x.
Proof.
apply: erefl.
Qed.
\end{lstlisting}

\subsection{Esercizio 3}

Si mostri che esistono i termini: $dec(z) \in Nat \quad [z \in N1 + Nat]$ e $cod(z) \in N1 + Nat \quad [z \in Nat]$ tali che per ogni numerale n: $dec(cod(0)) = 0$ e $dec(cod(n)) = n$

Nota: in Coq, il tipo induttivo \texttt{sum} (la somma disgiunta di due tipi A e B) è definito nel modo seguente:

\begin{lstlisting}[language=Coq]
Inductive sum (A B:Type) : Type :=
  | inl : A -> sum A B
  | inr : B -> sum A B.

Notation "x + y" := (sum x y) : type_scope.
\end{lstlisting}

Svolgimento dell'esercizio:

\begin{lstlisting}[language=Coq]
Definition dec (z : unit + nat) : nat :=
  match z with
  | inl tt => 0
  | inr n => n
  end.

Definition cod (z : nat) : unit + nat :=
  match z with
  | 0 => inl tt
  | S _ => inr z
  end.

Lemma dec_0 : dec (cod 0) = 0.
Proof.
apply: erefl.
Qed.

Lemma dec_n (n : nat) : dec (cod (S n)) = S n.
Proof.
apply: erefl.
Qed.
\end{lstlisting}
